\documentclass[xcolor=dvipsnames, aspectratio=169]{beamer}
\usepackage[ngerman]{babel}
\usepackage{amsmath}
\usepackage{amssymb}
\usepackage{amsfonts}
%\usepackage{mathpazo}
\usepackage{enumerate}
\usepackage{tikz}
\usepackage[version=4,arrows=pgf-filled]{mhchem}
\usepackage[retainorgcmds]{IEEEtrantools}
\usepackage{mathdots}
\usepackage{tcolorbox}
\usepackage{multirow}
\usepackage{booktabs}
\usepackage{tabularx}
\usetikzlibrary{matrix,backgrounds,patterns,arrows,decorations.markings,shapes,snakes}
\usepackage{caption}

\usetheme[titleformat title=regular,titleformat frame=regular,titleformat section=allcaps,numbering=fraction]{metropolis} 
%\setsansfont[BoldFont={Fira Sans SemiBold}]{Fira Sans Book}
%\setmonofont[Scale=1.1]{Ubuntu Mono}

\author[Felix Kußmaul]{\large Felix Kußmaul}
\title[NAA]{\Large Die Neutronenaktivierungsanalyse}
\subtitle{Einführung und Anwendungen in der Archäologie}
\institute[UzK]{Seminar zu naturwissenschaftlichen Untersuchungsmethoden\\ von Fundkeramik und ihrer archäologischen Interpretation\\[.5em] Universität zu Köln}
\date[23.\ Juli 2016]{23.\ Juli 2016}

\newcommand{\red}[1]{\textcolor{red}{#1}}
\newcommand{\orange}[1]{\textcolor{orange}{#1}}
\newcommand{\green}[1]{\textcolor{markgreen}{#1}}
\newcommand{\textgreen}[1]{\textcolor{textgreen}{#1}}
\newcommand{\blue}[1]{\textcolor{textblue}{#1}}
\newcommand{\gray}[1]{\textcolor{gray}{#1}}

\newcolumntype{R}{>{\centering\raggedleft\arraybackslash}X}
\newcolumntype{L}{>{\centering\raggedright\arraybackslash}X}
\newcolumntype{C}{>{\centering\arraybackslash}X}

\usepackage[firstinits=true,style=archaeologie,width=2cm,edby,backend=biber]{biblatex}
\addbibresource{naa.bib}
\renewcommand*{\bibfont}{\small}

\setbeamercovered{invisible}

\newcommand{\textsb}[1]{{\fontfamily{cmss}\fontseries{sbc}\fontshape{n}\selectfont #1}}

%COLORBOX
\newtcolorbox{mybox}[1]{
colback=chameleongreen2!30,
colbacktitle=chameleongreen1!50,
coltitle=black,
colframe=textgreen,
boxrule=1pt,
titlerule=0pt,
arc=10pt,
title={\strut\textcolor{textgreen}{\textbf{#1}}}
}
\newtcolorbox{mybluebox}[1]{
colback=chameleongreen4!30,
colbacktitle=chameleongreen4!50,
coltitle=black,
colframe=textblue,
boxrule=1pt,
titlerule=0pt,
arc=10pt,
title={\strut\textcolor{textblue}{\textbf{#1}}}
}

\setbeamertemplate{footline}
{
\hbox{%
  \begin{beamercolorbox}[wd=.28\paperwidth,ht=2.7ex,dp=1.2ex,center]{author in head/foot}%
    \usebeamerfont{author in head/foot}\insertshortauthor\ (\insertshortinstitute)
  \end{beamercolorbox}%
  \begin{beamercolorbox}[wd=.44\paperwidth,ht=2.7ex,dp=1.2ex,center]{author in head/foot}%
    \usebeamerfont{title in head/foot}\insertshorttitle:\ \textbf{\insertsection}
  \end{beamercolorbox}%
  \begin{beamercolorbox}[wd=.28\paperwidth,ht=2.7ex,dp=1.2ex,center]{author in head/foot}%
    \usebeamerfont{date in head/foot}\insertshortdate\hfill\insertframenumber/\inserttotalframenumber\strut
  \end{beamercolorbox}}
  \vskip0pt%
}

\tikzset{
  invisible/.style={opacity=0},
  visible on/.style={alt={#1{}{invisible}}},
  alt/.code args={<#1>#2#3}{%
    \alt<#1>{\pgfkeysalso{#2}}{\pgfkeysalso{#3}} % \pgfkeysalso doesn't change the path
  },
}

\newcommand{\printSectionYes}{\AtBeginSubsection[]
{
 \begin{frame}{Agenda}
 \tableofcontents[sectionstyle=show/shaded,
 					subsectionstyle=show/shaded/hide]
 \end{frame}
}}

\tikzset{onslide/.code args={<#1>#2}{%
  \only<#1>{\pgfkeysalso{#2}}%
}}

\setbeamertemplate{section in toc shaded}[default][50]

\makeatletter
\patchcmd{\beamer@sectionintoc}{\vskip1.5em}{\vskip0.5em}{}{}
\makeatother

\makeatletter
\newcommand\footnoteref[1]{\protected@xdef\@thefnmark{\ref{#1}}\@footnotemark}
\makeatother

\setbeamertemplate{bibliography item}{%
  \ifboolexpr{ test {\ifentrytype{book}} or test {\ifentrytype{mvbook}}
    or test {\ifentrytype{collection}} or test {\ifentrytype{mvcollection}}
    or test {\ifentrytype{reference}} or test {\ifentrytype{mvreference}} }
    {\setbeamertemplate{bibliography item}[book]}
    {\ifentrytype{online}
       {\setbeamertemplate{bibliography item}[online]}
       {\setbeamertemplate{bibliography item}[article]}}%
  \usebeamertemplate{bibliography item}}
  
\defbibenvironment{bibliography}
  {\list{}
     {\settowidth{\labelwidth}{\usebeamertemplate{bibliography item}}%
      \setlength{\leftmargin}{\labelwidth}%
      \setlength{\labelsep}{\biblabelsep}%
      \addtolength{\leftmargin}{\labelsep}%
      \setlength{\itemsep}{\bibitemsep}%
      \setlength{\parsep}{\bibparsep}}}
  {\endlist}
  {\item}
\pgfdeclareimage[width=\columnwidth]{snucleus}{snucleus.eps}
\definecolor{morange}{HTML}{FF8200}
 
\begin{document}

\maketitle

\begin{frame}[<+->]{Before we start \dots}
\begin{itemize}
\item Zu langsam/schnell/leise/laut/undeutlich? \textbf{Bitte meckern!}
\item Fragen \alert{jederzeit} willkommen
\item Ich bin kein Kernphysiker, deshalb:

Darstellungen und Ausführungen sind stark vereinfacht!
\item Du sollst keine anderen Editoren neben \texttt{vim} haben
\end{itemize}
\end{frame}

\begin{frame}{Agenda}
       \tableofcontents[ 
  		subsectionstyle=show, 
   	 	sectionstyle=show, 
   	 ] 
\end{frame}

\section{Einordnung}

\begin{frame}{Sinn und Zweck}
\begin{quote}
``NAA is a sensitive analytical technique useful for performing both qualitative and quantitative \alert{multi-element analysis} of major, minor, and trace elements in samples from almost every conceivable field of scientific or technical interest.''
\end{quote}
\vspace*{-2em}
\begin{flushright}
-- Dr.\ Michael Glascock, MURR
\end{flushright}
\end{frame}

\begin{frame}{\dots\ in der Archäologie}
Wir Archäologen nutzen NAA seit Jahrzehnten erfolgreich für
\begin{itemize}
\item die Bestimmung der \textbf{Herkunft} eines Objektes
\item die Überprüfung der \textbf{Echtheit} eines fragwürdigen Gegenstandes
\item Studien zur \textbf{Herstellung} und \textbf{Nutzung} von Artefakten
\end{itemize}
\end{frame}

\begin{frame}{Herkunftsbestimmung}
Wir treffen folgende wichtige Annahmen:
\begin{enumerate}[(i)]
\item Kein Handel mit Rohton
\item Bei hoher Präzision gilt ein Elementmuster als \alert{charakteristisch} für einen Ort
\item Gefäße mit gleichem Elementmuster stammen vom selben Ort
\end{enumerate}
\end{frame}

\begin{frame}{Herkunftsbestimmung}
\begin{center}\Large
\begin{tabularx}{.9\columnwidth}{rlCrl}
\multirow{2}{*}{\parbox[c]{4.5em}{\includegraphics[scale=.6]{pot}}}&&&&\multirow{2}{*}{\parbox[c]{4.5em}{\includegraphics[scale=.6]{pot}}}\\
&\vphantom{$\overset{?}{=}$}\only<2-3>{Herkunft$_A$&$\overset{?}{=}$&Herkunft$_B$}&\\
&\onslide<3>{Muster$_A$&$=$&Muster$_B$}&\\
\end{tabularx}
\end{center}
\end{frame}
\begin{frame}{Herkunftsbestimmung}
\begin{center}\Large
\begin{tabularx}{.9\columnwidth}{rlCrl}
\multirow{2}{*}{\parbox[c]{4.5em}{\includegraphics[scale=.6]{pot}}}&&&&\multirow{2}{*}{\parbox[c]{4.5em}{\includegraphics[scale=.6]{pot}}}\\
&\color{morange}\vphantom{$\overset{?}{=}$}Herkunft$_A$&\color{morange}\vspace*{-1em}$=$&\color{morange}Herkunft$_B$&\\
&Muster$_A$&$=$&Muster$_B$&\\
\end{tabularx}
\end{center}
\end{frame}

\begin{frame}{Herkunftsbestimmung}
\textbf{Bekannte Herausforderungen:}
\begin{itemize}[<+->]
\item Nur selten wurde Rohton direkt verwendet.

\textit{Meist wurde er vor dem Töpfern gesäubert oder mit anderem Ton vermischt.}
\item Variation bei der Aufbereitung führt zur Veränderung der Zusammensetzung.

\textit{Die Konzentrationswerte müssen relativ zueinander betrachtet werden.}
\item Veränderungen durch Brand und Bodenlagerung für bestimmte Elemente.

\textit{Die betroffenen Elemente werden bei der Messung nicht berücksichtigt.}
\item Messungenauigkeiten oder -fehler.
\end{itemize}
\end{frame}

\begin{frame}{Herkunftsbestimmung}
\textbf{Problem mit Streuungen}

Gewisse Streuungen (5-10\%) sind normal für Proben mit homogener Masse.

Zur Einteilung in Gruppen von Proben wollen wir nicht mehr als diesen Toleranzbereich zulassen, da sonst das zu \glqq diffuse\grqq\ Gesamtmuster eine Unterscheidung verschiedener Herkünfte stark erschwert.
\end{frame}

\begin{frame}{Zuordnung: Muster $\mapsto$ Werkstatt, VLLT STREICHEN?!}
\begin{itemize}
\item Schwerster Schritt.
\item Notwendig: Finden von Referenzmaterial \textit{für jede Produktionsserie}.

Dafür gut geeignet sind \alert{Fehlbrände} aus Abfallhaufen.\medskip\pause

Alternativ: Logische Schlüsse anhand von \alert{Verteilungen}:

Lokale Produktion genau dann, wenn chemisches Muster dort in großen Stückzahlen gefunden wurde (nicht nur Keramik)
\end{itemize}
\end{frame}

\begin{frame}[<+->]{Gruppenbildung (\emph{Clustering})}
\begin{enumerate}[(1)]
\item Jede Probe $s$ ist ein \alert{Punkt} in einem $n$-dimensionalen Raum: \[s=(c_1,c_2,\dots,c_n)\] wobei $c_i$ die Konzentration des $i$-ten Elements in der Probe ist.
\item Ähnlich zusammengesetzte Proben liegen in einer \alert{Punktwolke}.
\item Finden von Punktwolken durch statistische Verfahren am Computer

\textbf{$\Rightarrow$ äquivalent zum Gruppieren der Proben!}
\end{enumerate}
\end{frame}

\begin{frame}[<+->]{Gruppenbildung (\emph{Clustering}): Beispiel}
Ermitteln von zweidimensionalen Bildern durch \emph{Diskriminanzanalyse} (anm: zB 3 gruppen: mittelpunkte d. wolken spannen ebene auf)
\end{frame}

\section{Neutronenaktivierung}

\subsection{Grundlagen}\printSectionYes
\begin{frame}[<+->]{Grundbegriffe Kernphysik}
\begin{enumerate}[(i)]
\item \emph{Protonen} $p$ sind positiv geladene Teilchen mit Masse 1
\item \emph{Neutronen} $n$ sind el.\ neutral geladene Teilchen mit Masse 1
\item Protonen und Neutronen fassen wir als \emph{Nukleonen} zusammen
\item Protonenzahl (\emph{Ordnungszahl}) $Z$ im Atomkern bestimmt das \emph{Element}
\item Nukleonenzahl (\emph{Massenzahl}) $A$ im Atomkern bestimmt die \emph{Atommasse}
\item Wir schreiben: \[\text{\ce{^A_Z Element}\quad oder kurz \quad\ce{^A Element}}\]
\end{enumerate}
\end{frame}

\begin{frame}{Erklärung am Beispiel}

{\huge\begin{equation*}
\text{\ce{^{14}_{6} C}}
\end{equation*}}\bigskip

Kohlenstoffisotop mit $14$ Nukleonen im Kern: $6$ Protonen und $14-6=8$ Neutronen.
\end{frame}

\begin{frame}[<+->]{Grundbegriffe Radioaktivität}
\begin{enumerate}[(i)]
\item Atomkerne teilt man in zwei Gruppen auf:
\begin{enumerate}[(1)]
\item Stabile Nuklide $\Rightarrow$ zerfallen \emph{nicht} von selbst
\item Radionuklide $\Rightarrow$ instabil; Spontanzerfall, bis stabiler Zustand erreicht wird
\end{enumerate}
\item Produkte des Zerfalls bzw.\ der Spaltung:
\begin{enumerate}[(a)]
\item $\alpha$-Strahlung: Helium-Kerne (zwei Protonen, zwei Neutronen)
\item $\beta$-Strahlung: Elektronen bei $\beta^-$, Positronen bei $\beta^+$
\item $\gamma$-Strahlung: Photonen
\end{enumerate}
\end{enumerate}
\end{frame}

\subsection{Neutronenaktivierung}

\begin{frame}{Neutronenaktivierung}
\textbf{Prinzip:}
\begin{center}
Neutronen werden auf ein Objekt geschossen, dessen Atome dann\\ nach und nach angereichert und somit radioaktiv. Die Strahlung\\ gibt Aufschluss über die Elementkonzentration im Objekt.
\end{center}
\end{frame}

\begin{frame}{Neutronenaktivierung}
\begin{center}\vspace*{-2.5em}
\begin{tikzpicture}
\draw[color=white,line width=.01pt] (-7,-3) grid[xstep=7cm, ystep=3cm] (7,3);

\pgfbox[center,center]{\pgfuseimage{snucleus}};

\node at (-6.4,-1.1) {$n$};
\node at (-5.2,-3.5) {\Large \ce{^{A}_{Z} X}};
\node at (-2.3,-3.5) {\Large \ce{^{A+1}_{Z} X^*}};
\node at (.4,-3.5) {\Large \ce{^{A+1}_{Z} X}};
\node at (3.2,-3.5) {\Large \ce{^{A+1}_{Z±1} Y^*}};
\node at (6.1,-3.5) {\Large \ce{^{A+1}_{Z±1} Y}};
\node at (2.5,2) {\large $\beta^±$};
\node at (-0.7,-1.6) {\Large $\gamma$};
\node at (4.9,-1.6) {\Large $\gamma$};
\end{tikzpicture}
\end{center}
\end{frame}

\begin{frame}{Neutronenaktivierung}
\[\huge\text{\ce{X + n -> Y + \gamma}}\]
%\includegraphics[scale=1]{asd.png}
\end{frame}

\section{Anwendungsbeispiel}

\section{Einrichtungen}

\section{Fazit}
\begin{frame}[allowframebreaks]{Literatur}
\nocite{*}

\printbibliography[heading=none]
\end{frame}

\maketitle

\end{document}